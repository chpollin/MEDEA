\documentclass[12pt,a4paper]{article}
\usepackage[utf8]{inputenc}
\usepackage{setspace}
\usepackage[german]{babel}
\usepackage{graphicx} 
\usepackage[T1]{fontenc}
\usepackage{amsmath}
\usepackage{amsfonts}
\usepackage{amssymb}
\usepackage{url}
\usepackage[bf]{caption}
\usepackage[a4paper]{geometry}
\usepackage{float}
\usepackage{acronym}
\usepackage{pdfpages}
\usepackage{enumitem}


\usepackage{jurabib}

\jurabibsetup{
	commabeforerest,
	ibidem=strict,
	%here you can change full citation, none, first
	citefull=none,
	see,
	titleformat={colonsep,all},
}

\renewcommand*{\jbauthorfont}{\textsc}
\renewcommand*{\biblnfont}{\scshape\textbf}
\renewcommand*{\bibfnfont}{\normalfont\textbf}
\AddTo\bibsgerman{%
	\renewcommand*{\ibidemname}{ebd.}
	\renewcommand*{\ibidemmidname}{ebd.}
}

\usepackage[bottom,hang]{footmisc}
\setlength{\footnotemargin}{0pt}


% Für source codes etc.
\usepackage{listings}

\usepackage{color}
\definecolor{gray}{rgb}{0.4,0.4,0.4}
\definecolor{darkblue}{rgb}{0.0,0.0,0.6}
\definecolor{cyan}{rgb}{0.0,0.6,0.6}

\lstset{
  numbers=left,   
  basicstyle=\ttfamily,
  columns=fullflexible,
  showstringspaces=false,
  commentstyle=\color{gray}\upshape
}

\lstdefinelanguage{XML}
{
  basicstyle=\footnotesize\ttfamily,
  morestring=[b]",
  morestring=[s]{>}{<},
  morecomment=[s]{<?}{?>},
  stringstyle=\color{black},
  identifierstyle=\color{darkblue},
  keywordstyle=\color{cyan},
  morekeywords={xmlns,version, type, about, resource, lang}
  % list your attributes here
}

\geometry{a4paper,left=25mm,right=25mm, top=25mm, bottom=25mm}
\onehalfspacing
\renewcommand{\familydefault}{ptm}

\usepackage[bottom,hang]{footmisc}
\usepackage{fnpct}


\setlength{\footnotemargin}{0pt}

\setlength{\parindent}{0pt}


\author{Christopher Pollin}
\title{Exposé\\ \textbf{Formale, digitale Methoden in den Geschichtswissenschaften. Am Beispiel digital edierter Rechnungsbücher}}
\date{\today{}, Graz}
\begin{document}
\pagenumbering{gobble}
\AdaptNoteOpt\footcite\multfootcite
\maketitle
\tableofcontents

\newpage
\pagenumbering{arabic}

\section{Einleitung}

In meiner Masterarbeit meines ersten Studiums (EuroMACHS) habe ich mich mit der Fragestellung auseinandergesetzt wie mittels Technologien des \textit{Web of Data (Semantic Web)} Information Retrieval und Resource Discovery für eine digitale Sammlung umgesetzt werden kann. Dies umfasste einen Teilbestand des Kriminalmuseums der Universität Graz, der in der Langzeit-Infrastruktur GAMS veröffentlicht wurde. Um mein zweites Masterstudium (Geschichte) abzuschließen, soll sich die dafür notwendige Masterarbeit mit der Theorie und Anwendung formaler und digitaler Arbeitstechniken im Fachbereich Geschichte beschäftigen. Da ich in einem Projekt zur semantischen Anreicherung von digital edierten historischen Rechnungsbüchern in der technischen Umsetzung und Datenmodellierung angestellt bin bietet sich dieses Thema für eine Abschlussarbeit im Modul Historische Fachinfomratik an.
\\
Am Beginn dieses Exposés steht eine Spezifizierung der Thematik und Relevanz des Themas, sowie eine daraus abgeleitete Forschungsfrage. Nach einem Überblick über wichtige Literatur, Projekte und einer Skizze des Aufbaus der Arbeit, folgt eine Erörterung der Methoden und der Umsetzung. Am Ende werden erwartete Ergebnisse und Herausforderungen reflektiert.

\section{Thematik, Projektkontext und Forschungsfrage}

Historische Rechnungsbücher liefern reichhaltige und strukturierte Datensätze, die oft längere Zeiträume abdecken und die als Aggregation vieler Einzelinformationen enthalten zu Beantwortung unterschiedlicher Forschungsfragen herangezogen werden können. Eine Transkription allein reicht nicht aus um die unterschiedlichen Dimensionen einer solchen Quelle abzudecken: die linguistische/textuelle, die quantifizierbare und die semantische Dimension. Für Forschungszwecke unterliegen historische Quellen einem Transformationsprozess hin zu (vernetzten) Informationsquellen, die in verschiedenen Forschungsszenarien genutzt werden können. Um dies zu veranschaulichen, werden drei Fallstudien von Projektpartnern und ihren jeweiligen Forschungsinteressen diskutiert, die weit über wirtschaftliche und administrative Aspekte hinausgehen.
\\
Die 3 Beispiele.
\\
\\
Der \textbf{Projektkontext} in dem ich angestellt bin, ist eine durch die \textit{Andrew W. Mellon Foundation} geförderte und durch das \textit{Wheaton College Massachusetts} koordinierte Kooperation des Zentrums für Informationsmodellierung mit Partnern aus den USA. Es verfolgt das Ziel semantisch angereicherte digitale Editionen von historischen Rechnungsbüchern einem breiten Fachpublikum zugänglich zu machen. Daten - aus unterschiedlichen Formaten - sollen auf einer gemeinsamen Plattform zusammengeführt werden und adäquate Formen des Retrievals, Discoverys und der Visualisierung eröffnen, um die Arbeit mit den Quellen zu erleichtern. Die Überführung nach RDF auf Basis der im Projektkontext entwickelten \textit{Bookkeeping-Ontologie}, die Transferprozesse historischer Rechnungsbücher formalisiert, erlaubt die Interoperabilität, Verlinkung und Zusammenführung der Informationen im Sinne des \textit{Web of Data} und \textit{Linked Open Data}.
\\
\\
Ziel der Masterarbeit soll es nicht sein die Projektinhalte zu dokumentieren, sondern sich mit theoretischen und praktischen Fragestellungen zu formaler Modellen und formaler Methoden in den Geschichtswissenschaften  auseinanderzusetzen, wiewohl Quellen, Workflows und Daten aus dem DEPCHA Projekt einfließen sollen.

\section{Projekte, Basisliteratur und Aufbau}

Ein \textbf{Projekt} in diesem Zusammenhang ist \textit{ResearchSpace}\footcite{oldman2018reshaping} am British Museum. Es ist ein Open Source Pilotprojekt zur Etablierung einer kollaborativen \textit{Web of Data} Umgebung, die dabei helfen soll, Information über geisteswissenschaftliche Domänen hinweg nutzbar zu machen. Es wird versucht sowohl auf konzeptioneller Ebene, als auch auf Ebene der Daten, Interoperabilität zu schaffen, indem Entitäten wie Personen, Orte oder Konzepte verknüpft werden.
Das \textit{Data for History Consortium}\footcite{beretta2017dataforhistory} geht einen vergleichbaren Weg und versucht ein gemeinsames Set an Methoden im \textit{Web of Data} zu entwickeln, um Daten in den Geschichtswissenschaften zu modellieren, verknüpfen und auszutauschen.
In der Archivwelt wird 2019 der Standard \textit{Records in Context}\footcite{llanes2017records}
veröffentlicht, der es erlauben soll Archivalien und ihre multikontextuellen Dimensionen zu beschreiben.
\textit{OLia}\footcite{chiarcos2015olia} repräsentiert ein Repositorium für Terminologien für linguistische Phänomene und dient der Vereinheitlichung von Annotaionsmodellen für Textkorpora in den Sprachwissenschaften. 
Die soeben genannten Projekte tragen zur Etablierung des \textit{Web of Data} in den einzelnen Fachdomänen bei und eröffnen neue Möglichkeiten für die digitalen Geisteswissenschaften.
\\
\\
Am Anfang der Arbeit steht neben einer allgemeinen theoretischen Diskussion zu klassischen Themen der Informationswissenschaft (Daten - Information - Wissen)\footcite{favre2001information}, eine  Auseinandersetzung auf konzeptioneller\footcite{berners2001semantic}\footcite{cardoso2007semantic} und technischer\footcite{bernstein2016new} Ebene mit dem \textit{Web of Data}, wobei ein Fokus auf Wissensmodellierung\footcite{kelly2016practical} , Ontologien\footcite{stuckenschmidt2009ontologien} und \textit{Linked Open Data}\footcite{rietveld2015linked}\footcite{bauer2011linked} liegt, sowie eine kritischer Auseinandersetzung mit dem \textit{Web of Data}\footcite{swartz2013aaron}.
Im nächsten Kapitel wird auf Forschungsdaten\footcite{andorfer2015forschungsdaten}\footcite{cremer2018chimare} und digitale Forschungsinfrastruktur\footcite{burghardt2015informationsinfrastruktur}\footcite{neuroth2016nachhaltigkeit} , am Beispiel von drei auf semantischen Technologien beruhenden Projekten (SZD, DEPCHA, AAIF), eingegangen. Dies beinhaltet die Forschungsziele, konzeptionelle Ebene (Ontologien) und Forschungsdaten (RDF, GAMS) dieser Projekte, auf die später noch weiter eingegangen wird.
\\
\\
Auf diesen Grundlagen und den Praxisbeispielen wird herausgearbeitet, dass es sich beim Web of Data nicht nur um einen Technologie Stack handelt, sondern um eine eigene Methode um wissenschaftliche Arbeit und Dokumentation\footcite{latour2016zirkulierende} zu befördern. Dazu werden zuerst formale Methoden in den Geisteswissenschaften diskutiert und anschließend um digitale Methoden erweitert.\footcite{thaller2017digital}\footcite{thaller2017ungefahre} Das Ontology Engineering\footcite{hitzler2016ontology} , Informationsvisualisierung\footcite{rehbein2017informationsvisualisierung}\footcite{jager2015informationsvisualisierung} , Information Retrieval\footcite{baezayates2011retrieval} , Resource Discovery\footcite{wiesenmueller2016resourcediscovery} und das Reasoning\footcite{bursztyn2015reasoning} spielen dabei eine hervorgehobene Rolle.
Da die Entwicklung eines \textit{Proof of Concept} angedacht ist, widmet sich ein Kapitel der Entwicklung, dem Design\footcite{shneiderman2016designing} und der Evaluierung\footcite{buttcher2016information} von Suchstrategien im Kontext wissenschaftlicher Informationssysteme.\footcite{Khalili2016Adaptive} Theoretische sowie praktische Beispiele werden diskutiert. Ein Schwerpunkt liegt auf informations- bzw. wissensbasiertem Information Retrieval und Resource Discovery\footcite{GarciaKMD13}\footcite{DamianoLL14} .
Abschließend steht eine Dokumentation der Implementierung, der Evaluierung und Reflexion der Umsetzung in der GAMS-Infrastruktur\footcite{stigler2018gams} der angeführten Projekten.

\section{Method und Umsetzung}


Im Zuge des Projektes \textbf{Digital Edition Publishing Cooperative for Historical Accounts (DEPCHA)}\footnote{\url{gams.uni-graz.at/depcha}} wird ein gemeinsamer Publikations-Hub für historische Rechnungsbücher umgesetzt. Im Zentrum steht die Entwicklung und Nutzung einer Ontologie zur Formalisierung und Standardisierung von Buchungstransaktionen in historischen Rechnungsunterlagen, sowie ein \textit{Linked Open Data} Zugang. Der Mehrwert entsteht durch Funktionalitäten des Retrievals, der Visualisierung und Analyse der eingespielten Datenbestände.
\\
Für alle diese Projekte existieren hochstrukturierte RDF-Daten, die jeweils mit einer domänenspezifischen Ontologie beschrieben sind. Diese Ontologie wurde in einem iterativen \textit{Ontology Enginneering}-Prozess mit den FachkollegInnen, basierend auf den fachspezifischen Forschungsfragen, generiert.

The work of historians is an interpretation of relics from the past. Therefore, when using formal methods on historical data, research should distinguish between the representation of the original source and its interpretation. The latter is the core knowledge domain of historical research. It is advised to share the basic assumptions and definitions in a knowledge domain in a formal way. Linked Open Data is a central approach to this. This particularly applies to historical accounts, which provide large, highly structured data sets over long timespans, if the individual information entities are prepared for formal analysis.

DEPCHA, a Mellon funded cooperation of five US partners and the Centre for Information Modelling at Graz University, aims to link the knowledge domain of economic activities to historical accounting records. It creates a publication hub for digital editions, converting multiple formats (CSV, TEI/XML) into RDF and publishing these alongside the transcriptions. The DEPCHA prototype is based on the long term preservation oriented GAMS infrastructure.

The common knowledge domain of these documents is formalized in a ”bookkeeping” ontology, based on the REA model and compliant with the CIDOC CRM. As a conceptual data model, the ontology is developed in an iterative process. It formalizes the interpretation of transactions of money, commodities and services from one actor to another, and further properties that can be found in historical accounts. The RDF data extracted from the accounts becomes therefore a highly structured and self describing data set, being interoperable and reusable for researchers in diverse fields. The RDF representation can link to URI’s of commodities, places, persons or other LOD vocabularies. Additionally the RDF representation contributes to the LOD. Thus, all formal methods applied in the DEPCHA project can be transferred to other data conforming to the proposed ontology and any kind of combined data set.



\newpage
\nocite{*}
\bibliographystyle{jurabib}
\bibliography{literatur}
\newpage

%
%\subsection{Technologien}
%\begin{table}[H]
%\begin{tabular}{|l|l|lll}
%\cline{1-2}
% GAMS&a  &  &  &  \\ \cline{1-2}
% RDF&b  &  &  &  \\ \cline{1-2}
% RDFs&  &  &  &  \\ \cline{1-2}
% OWl&  &  &  &  \\ \cline{1-2}
% SPARQL&  &  &  &  \\ \cline{1-2}
% Apache Jena&  &  &  &  \\ \cline{1-2}
% AngularJs&  &  &  &  \\ \cline{1-2}
% Ajax&  &  &  &  \\ \cline{1-2}
% Bootstrap 4&  &  &  &  \\ \cline{1-2}
%\end{tabular}
%\caption{My caption}
%\label{my-label}
%\end{table}
%
%\subsection{Standards}
%\begin{table}[H]
%\begin{tabular}{|l|l|lll}
%\cline{1-2}
%CIDOC&a  &  &  &  \\ \cline{1-2}
%TEI&b  &  &  &  \\ \cline{1-2}
%Records in Context&  &  &  &  \\ \cline{1-2}
%OLiga&  &  &  &  \\ \cline{1-2}
%&  &  &  &  \\ \cline{1-2}
%&  &  &  &  \\ \cline{1-2}
%&  &  &  &  \\ \cline{1-2}
%&  &  &  &  \\ \cline{1-2}
%&  &  &  &  \\ \cline{1-2}
%\end{tabular}
%\caption{My caption}
%\label{my-label}
%\end{table}
\renewcommand{\labelenumii}{\arabic{enumi}.\arabic{enumii}}
\newpage
\section{Gliederung}
\begin{enumerate}
 \item Das Web of Data 
   \begin{enumerate}
   \item  Schnittpunkt Informationswissenschaft und Geisteswissenschaft
    \item Konzepte und Vision
    \item Technische Grundlagen
    \item Konzeptionelle Ebene: Ontologie
    \item Datenebene: Linked Open Data
    \item Web of Data und die digitalen Geisteswissenschaften
    \item Kritik und Alternativen
  \end{enumerate}
\item (Digitale) Geisteswissenschaften und Forschungsdaten
\begin{enumerate}
	\item Warum digitale Geisteswissenschaften?
	\item Forschungsdaten und Infrastruktur
    \item Die virtuelle Nachlassrekonstruktion Stefan Zweig Digital - SZD 
    \begin{enumerate}
    	\item Nachlassontologie und Records in Context
     \end{enumerate}
    \item Digital Edition Publishing Cooperative for Historical Accounts - DEPCHA
    \begin{enumerate}
    	\item Bookkeeping Ontology und Data for History
     \end{enumerate}
    \item Open Access Database 'Adjective-Adverb Interfaces in Romance' - AAIF
    \begin{enumerate}
    	\item AAIF-Annotationsmodell und OLiA Ontologies
     \end{enumerate}
  \end{enumerate}
  \item Web of Data als Methode
  \begin{enumerate}
    \item Formale Methoden in den Geisteswissenschaften
    \item Daten - Information - Wissen
    \item Ontology Engineering
    \item Deskriptive Logik und Reasoning
    \item Information Retrieval 
    \item Resource Discovery
    \item Informationsvisualisierung
  \end{enumerate}
  \item Informationsbasiertes Information Retrieval und Resource Discovery  
   \begin{enumerate}
    \item Best Pratice Beispiele
    \item Anforderungen an geisteswissenschaftlichen Projektkontexten
    \item Design 
    \item Umsetzungen
    \item Evaluierungsmethoden
  \end{enumerate}
  \item Umsetzung: GAMSKitz - Knowledge-based informaTion visualization
  \begin{enumerate}
  	\item Technologie Stack
    \item GAMS
    \item Query-Objekt
    \item Ontologie und Search Interface
    \item Ontologie und Informationsvisualisierung 
    \item Umsetzung für SZD, DEPCHA und AAIF 
  \end{enumerate}
  \item Conclusio und Ergebnisse
\end{enumerate}
\newpage
\section{Arbeitsplan}

\begin{itemize}
\item 01.2017: Projektbeginn SZD
\item 08.2017: Projektbeginn DEPCHA
\item 01.2018: Projektbeginn AAIF
\item 03.2018: Beginn Doktorat
\item 12.2018: Publikation SZD, DEPCHA, AAIF
\item 01.2019: Exposé
\item 06.2019: Ontologie und RDF Daten für SZD, DEPCHA, AAIF
\item 09.2019: Leseliste
\item 11.2019: Kapitel - \textit{Web of Data Vision, Konzepte und Grundlagen} 
\item 03.2020: Kapitel - \textit{(Digitale) Geisteswissenschaften und Forschungsdaten}
\item 09.2020: Prototype auf lokaler GAMS
\item 12.2020: Evaluierung der Umsetzung
\item 10.2020: Kapitel - \textit{Web of Data als Methode}
\item 02.2021: Kapitel - \textit{ Informationsbasiertes Information Retrieval und Resource Discovery  }
\item 06.2021: Umsetzung auf GAMS
\item 08.2021: Kapitel - \textit{Umsetzung}
\item 12.2021: Evaluierung der Umsetzung
\item 02.2021: Kapitel - \textit{Conclusio}
\item 04.2022: Abgabe
\end{itemize}


\end{document}
