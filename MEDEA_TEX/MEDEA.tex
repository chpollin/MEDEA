\documentclass[12pt,a4paper]{article}
\usepackage[utf8]{inputenc}
\usepackage{setspace}
\usepackage[german]{babel}
\usepackage{graphicx} 
\usepackage[T1]{fontenc}
\usepackage{amsmath}
\usepackage{amsfonts}
\usepackage{amssymb}
\usepackage{url}
\usepackage[bf]{caption}
\usepackage[a4paper]{geometry}
\usepackage{float}
\usepackage{acronym}
\usepackage{pdfpages}


\usepackage{jurabib}

\jurabibsetup{
	commabeforerest,
	ibidem=strict,
	citefull=first,
	see,
	titleformat={colonsep,all},
}

\renewcommand*{\jbauthorfont}{\textsc}
\renewcommand*{\biblnfont}{\scshape\textbf}
\renewcommand*{\bibfnfont}{\normalfont\textbf}
\AddTo\bibsgerman{%
	\renewcommand*{\ibidemname}{ebd.}
	\renewcommand*{\ibidemmidname}{ebd.}
}

\usepackage[bottom,hang]{footmisc}
\setlength{\footnotemargin}{0pt}

\geometry{a4paper,left=25mm,right=25mm, top=25mm, bottom=25mm}

% Für source codes etc.
\usepackage{listings}

\usepackage{color}
\definecolor{gray}{rgb}{0.4,0.4,0.4}
\definecolor{darkblue}{rgb}{0.0,0.0,0.6}
\definecolor{cyan}{rgb}{0.0,0.6,0.6}

\lstset{
  numbers=left,   
  basicstyle=\ttfamily,
  columns=fullflexible,
  showstringspaces=false,
  commentstyle=\color{gray}\upshape
}

\lstdefinelanguage{XML}
{
  basicstyle=\footnotesize\ttfamily,
  morestring=[b]",
  morestring=[s]{>}{<},
  morecomment=[s]{<?}{?>},
  stringstyle=\color{black},
  identifierstyle=\color{darkblue},
  keywordstyle=\color{cyan},
  morekeywords={xmlns,version, type, about, resource, lang}
  % list your attributes here
}

\geometry{a4paper,left=25mm,right=25mm, top=25mm, bottom=25mm}
\onehalfspacing
\renewcommand{\familydefault}{ptm}

\usepackage[bottom,hang]{footmisc}
\setlength{\footnotemargin}{0pt}

\setlength{\parindent}{0pt}


\author{Christopher Pollin}
\title{MEDEA - Modellierung semantisch angereicherter digitaler Edition von historischen Rechnungsbüchern}
\date{\today{}, Graz}
\begin{document}
\pagenumbering{gobble}

\maketitle
\tableofcontents

\newpage
\pagenumbering{arabic}

\section{Einleitung}
\label{sec:Einleitung}

\subsection{Aufbau der Arbeit}
\label{subsec:Aufbau}

\subsection{Literaturüberblick}
\label{subsec:Literaturueberblick}

\section{Digitale Edition von historischen Rechnungsbücher}
\label{subsec:DigEdHiRe}

TOMASEK und BAUMAN beschreiben ein Modell eines interpretativen Markups, um Beziehungen zwischen Individueen, Geld- Güter- und Dienstleistungstransfer, die Doppeleintrag-Buchhaltung umfassenm auszuzeichnen. Die Auszeichnung basiert auf den ausdrucksstarken Richtlinien der TEI. \footcite[Vgl.][S.1-2, \protect\url{http://journals.openedition.org/jtei/895}, 08.03.2018]{tomasek2013encoding}

Rechungsbücher als Primärquelle haben eine lange Tradition, Dopplebuchung, bis Excel

Rechnungsbücher wie, die der Stadt basel sind tabellarisch angelegt, doch gibt es auch Quellen, wie etwa eine handgeschrieben Quittung für die Buchung eines Zimmers, in der die Semantik des Transfers im Text liegt. TEI eignet sich, so der Zugang von TOMASEK und BAUMAN, um solche Quellen mittels TEI auszuzeichnen.
Rechnungsbücher sind nicht unbedingt tabellarisch. [S.3-4]

Einfache Transkription im Sinne von Text und Zahlen in tabellarischer Form.


One research corpus is 

there might be methods of markup for HFRs that would take advantage of the longstanding existence of TEI standards for markup of text, both print and manuscript, to promote creation of large corpora of accessible digital versions of HFRs that might be used for historical research. [S.4-5]

Double-entry bookkeeping uses a specialized vocabulary, a professional jargon in which
the terms 'debtor' and 'creditor' have particular meanings. As a system, it models a set of
relationships between transactions recorded in the journal and accounts kept in a
separate ledger. Both Paciolo and Mair considered two separate books essential for
keeping double entry accounts: the daybook or journal and the ledger. Transactions
between a local businessman and his customer were recorded chronologically in the
journal, with a reference in the far left column to the page in the ledger where the
businessman recorded the customer's ongoing debt and the times when the debt was
settled. Thus, we see from this page in Laban Morey Wheaton's daybook that on October
2, 1849, George W. Braman purchased three cords of hardwood for \$12.75 and one of small
hardwood for \$2.00, for a total debit of \$14.75. 

This transaction was recorded as a credit to the debit side of Braman's account in
Wheaton's ledger. That is, Wheaton recorded that he had extended credit to Braman in
the amount of 14.75, to be paid at a later date, and we see that Braman settled his
account with Wheaton on January 1, 1850.
19 In this way, Braman's account demonstrates Wheaton's adherence to the principles of
double entry accounting as prescribed by Pacioli and Mair. Other accounts in Wheaton's
daybook and ledger do not match up so neatly, and Wheaton is by no means the only
merchant who kept less than perfect books. The popularity of double entry accounting
could not guarantee perfect adherence to its principles. Human use of standards of any
sort is subject to human error and idiosyncrasy. Nevertheless, use of double-entry
accounting has persisted over several centuries, so developing guidelines for its markup
seems prudent. [S.9-10]


\subsection{Single-Entry-System}

\subsection{Double-Entry-System}
\url{http://flegesunde.com/category/miscellanea/die-geschichte-der-doppelten-buchfuhrung.php}


\subsection{Transaction Model}
Transactions consist of main
components, which may be summarized as 'what', 'from whom', and 'to whom'.
TOMASEK und BAUMAN are pointing out four categories of transaction:
\begin{itemize}
\item A standard exchange Two transfers exists, where money or goods are purchased from one account to another. Like Christopher gives Geog 1 Euro and revieves an apple.
\item A barter, the exchange of goods without money. Like Chrisopher gives Georg 1 Cow and recieves an apple
\item A gift, which is a uni-directional transfer. Christopher gives Georg an apple
\item. A set of transfers between more thant two entities., as a multilateral trade.
\end{itemize}
\footcite[][p.13-15]{tomasek2013encoding}
\newpage
\section{Anhang}
\label{sec:Anhang}


\newpage

\subsection{Onlineressourcen}
\label{subsec:Onlineressourcen}

\begin{singlespace}


\textbf{Blazegraph}, \url{https://www.blazegraph.com}, 31.12.2016.
\\


\textbf{Dublin Core}, \url{http://dublincore.org}, 11.12.2016.
\\

\textbf{FEDORA}, \url{http://fedorarepository.org/about}, 05.02.2016.
\\

\textbf{GAMS}, \url{gams.uni-graz.at}, 05.02.2016.
\\

\textbf{GAMS Dokumentation}, \url{http://gams.uni-graz.at/docs}, 05.02.2016.
\\

\textbf{Interoperabilität},\\ \url{https://www.w3.org/blog/2008/05/open-standards-interoperability}, 12.04.2017.
\\

\textbf{Ontology}, \url{https://www.w3.org/standards/semanticweb/ontology}, 05.02.2016.
\\

\textbf{Paragraphen}, \url{gams.uni-graz.at/o:km.paragraphen/TEI_SOURCE}, 30.12.2016.
\\

\textbf{RDF}, \url{https://www.w3.org/RDF}, 20.12.2016.
\\

\textbf{Sesame}, \url{http://rdf4j.org/}, 05.02.2017.
\\

\textbf{SKOS}, \url{https://www.w3.org/2004/02/skos}, 05.02.2017. 
\\

\textbf{SPARQL}, \url{https://www.w3.org/TR/rdf-sparql-query}, 05.02.2016.
\\

\textbf{TEI}, \url{http://www.tei-c.org/index.xml}, 05.02.2016.
\\

\textbf{URI}, \url{https://de.wikipedia.org/wiki/Uniform_Resource_Identifier}, 04-03.2017.
\\

\textbf{W3C}, \url{https://www.w3.org/}, 05.03.2017.
\\

\textbf{XML}, \url{https://www.w3.org/XML}, 27.12.2016.
\\

\textbf{XSL-T}, \url{https://www.w3.org/TR/xslt}, 05.02.2016.
\\

\end{singlespace}

 
\newpage
\bibliographystyle{jurabib}
\bibliography{literatur}
\newpage
\listoffigures


\end{document}
